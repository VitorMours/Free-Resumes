\documentclass[12pt, a4paper]{paper}
\usepackage{babel}
\usepackage{graphicx}
\usepackage[margin=2.5cm]{geometry}
%\hfuzz=5.002pt 

\title{SQLAlchemy\\
  \hfill\includegraphics[height=3cm]{../../../images/universidade.png}
  \vspace{-3cm}
}
\subtitle{um tour pelo ORM mais utilizado dentro do ecossistema\newline python}
\author{João Vitor Rezende Moura - 1221184773}
\institution{UNIT - Universidade Tiradentes}

\begin{document}
\maketitle

\hrule
\begin{abstract}
  abstract
\end{abstract}
\vspace{-0.6cm} % vspace 
\begin{keywords}
  keywords
\end{keywords}
\hrule

\section{SQL e Dados Relacionais.} % (fold)
\label{sec:SQL e Dados Relacionais.}

SQL (Structured Query Language) é uma linguagem para fazer consultas em bancos de dados
depermite que nós façamos consultas dos dados presentes nesses bancos e nas tabelas 
que estão presentes dentro desses bancos de dados. Ela é relativamente semelhante entre 
os \textbf{Sistemas de Gerenciamento de Bancos de Dados(SGBDs)}, como: Oracle, MySQL, 
MariaDB e outros. 

Cada um deles entretanto, possui sua característica específica e podem ou não possuir 
um certo nível de compatibilidade com os outros sistemas. O SQL é uma forma padronizada 
do uso, criação, e manejamento dos dados, e ele permite uma maior facilidade também da 
integridade dos dados e facilita como a consulta desses addos pode ser feita. 

O SQL é utilizado em diversas áreas como uma forma de armazenar os dados de forma 
organizada quando são colhidos e tratados, algumas das áreas as quais o SQL e os SGBDs 
são: 

\begin{itemize}
  \item Desenvolvimento de Aplicações Web
  \item Análise de Dados e BI 
  \item Ciência de Dados e mineração de Dados 
  \item Aplicações Móveis 
  \item Educação e Pesquisa 
  \item Gestão de Finanças e Contabilidade
\end{itemize}

\subsection{Componentes de um sistema SQL} % (fold)
\label{sub:Componentes de um sistema SQL}
Um sistema SQL é composto por vários componentes para permitir a criação, manutenção e 
consulta de bancos de dados relacionais, esses componentes são: 

\begin{itemize}
  \item \textbf{Tabela: }As tabelas são estruturas fundamentais, pois elas funcionam como "pools" de dados, ou seja, armazenam dados em um grupo que é feito de formato tabular, com colunas representado atributos e linhas representando registros.
  \item \textbf{Procedimentos Armazenados: } Os procedimentos armazenados são blocos de código SQL que podem ser definidos e armazenados no banco de dados. Eles são usados para realizar tarefas específicas com cálculos complexos, validações de dados e processamentos personalizados.
  \item \textbf{Instruções: }As instruções servem para manipular os dados, e os comandos SQL permitem que os usuários realizem diversas ações, conforme as ações, e especifiquem a forma como os dados são manipulados por meio do conjunto dessas instruções por meio de determinadas lógicas.
\end{itemize}
% subsection Componentes de um sistema SQL (end)
% section SQL e Dados Relacionais. (end)

\newpage
\section{Introdução ao SQLAlchemy} % (fold)
\label{sec:Introdução ao SQLAlchemy}
\subsection{Do motor, ao mapa} % (fold)
\label{sub:Do motor, ao mapa}

  O SQLAlchemy é um conjunto de ferramentas SQL e um ORM que podem ser utilizados para 
trabalhar com bases de dados dentro do python. Seus componentes são divididos em duas 
estruturas primárias, que é o \textbf{core}, onde fica a parte mais direcionada à 
comunicação e integração com a base de dados, e o \textbf{orm}, onde fica a parte dos 
modelos e das abstrações de dados que podem ser usados para enviar os dados de uma 
determinada forma ao banco de dados.

Por definição técnica, temos que o \textbf{SQLAlchemy Core} é a parte direcionada à 
arquitetura fundacional do SQLAlchemy, como um conjunto de ferramentas para a base de 
dados. Ela provê ferramentas que são direcionadas à facilitar a integração com a base 
de dados, como: ferramentas para gerenciamento de conectividade, interatividade com as 
queries da base de dados, e permitir a construção de queries SQL diretamente. 

Por outro lado, o \textbf{SQLAlchemy ORM} é a parte direcionada ao 
\textbf{object relational mapping}, ou seja, provê técnicas dentro do sistema para 
facilitar a comunicação entre o banco de dados relacional e a linguagem de programação 
por meio de um "dialeto em comum". O ORM provê também uma camada adicional de 
configurações que permitem a definição de classes para serem mapeadas dentro das 
tabelas da base de dados, também como o mecanismo de \textbf{sessão}, que permite 
a persistência de objetos.

Podemos dizer com isso então, que o SQLAlchemy possui duas partes específicas, o motor 
de funcionamento que possibilita sua comunicação com o banco de dados, e o mapa, que 
permite a comunicação dos objetos de forma relacional para serem inseridos dentro do 
banco de dados.
% subsection Do motor, ao mapa (end)

\subsection{Estabelecendo Conectividade} % (fold)
\label{sub:Estabelecendo Conectividade}
A conexão do SQLAlchemy e do banco de dados é feito por meio da \textbf{engine}, e esse 
objeto atua como o a fonte da conexão com o banco de dados em específico. Ela 
normalmente é um objeto global que é criado somente uma vez em particular para a 
conexão com o servidor da base de dados, e a sua URL é configurada usando string que 
descrevem como a conexão deve ocorrer. Um exemplo de como isso ocorre, é mostrado a 
seguir numa conexão com um banco de dados sqlite3: 

\begin{verbatim}
  from sqlalchemy import create_engine
  engine = create_engine("sqlite+pysqlite:///:memory:", echo="True")
\end{verbatim}

A string que é passada dentro do método \texttt{create\_engine} permite o entendimento 
de que tipo de base de dados ele está se comunicando, qual driver do SQLAlchemy deve 
ser usado para fazer a conexão, que pode ser referenciada como \textbf{DBAPI}. E 
permite específicarmos a localização exata de onde fica a base de dados, que nesse 
caso especificamos que deve ficar dentro da memória. 

\paragraph{OBS:} % (fold)
\label{par:OBS:}
O método usado não inicia de fato a conexão com a base de dados a partir do momento o 
qual ela é criada, ela só é feita a partir do momento em que alguma tarefa lhe é 
ordenada, usando por tanto o padrão de inicialização tardia.
% paragraph OBS: (end)

\subsection{Usando a conexão} % (fold)
\label{sub:Usando a conexão}
O objetivo da engine  é exclusivamente para o usuário, fornecer a conectividade para 
a base de dados. Essa conexão é o objeto que podemos usar para interagir com a base e podemos fazer isso com o gerenciador de contexto, como na seguinte forma:

\begin{verbatim}
  from sqlalchemy import text 

  with engine.connect() as conn:
    result = conn.execute(text("select 'hello world'"))
    print(result.all())
\end{verbatim}

\subsection{Commitando mudanças} % (fold)
\label{sub:Commitando mudanças}

% subsection Commitando mudanças (end)







% subsection Estabelecendo Conectividade (end)

% section Introdução ao SQLAlchemy (end)





\end{document}

