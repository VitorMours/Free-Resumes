% File configurations
\documentclass[12pt, a4paper]{paper}
\usepackage{babel}
\usepackage{graphicx}
\usepackage[margin=2.5cm]{geometry}
\usepackage{listings}
\usepackage{multicol}
\title{Javascript \\
  \hfill\includegraphics[height=3cm]{../images/universidade.png}
  \vspace{-3cm}
}
\subtitle{Do básico ao avançado seguindo a MDN web docs}
\author{João Vitor Rezende Moura - 1221184773}
\institution{UNIT - Universidade Tiradentes}

% Specific elements created


\begin{document}
% Document specific elements
\maketitle

\hrule
\begin{abstract}
  abstract
\end{abstract}
\vspace{-0.6cm} % vspace 
\begin{keywords}
  Javascript, Front-End, UI, media-queries, web
\end{keywords}
\hrule

% ------------------ Content -------------------------------
\section{Primeiros passos com javascript} % (fold)
\label{sec:Primeiros passos com javascript}

% section Primeiros passos com javascript (end)

\section{Criando elementos em javascript} % (fold)
\label{sec:Criando elementos em javascript}

\subsection{Condicionais dentro do javascript} % (fold)
\label{sub:Condicionais dentro do javascript}
Dentro de qualquer linguagem de programação, temos que tomar decisões, 
sejam elas baseadas nas informações providas pelo usuário, nas informações 
que conseguimos captar dele, ou seja por algum outro motivo. Esse tipo de 
escolha, pode ser feito por meio de uma estrutura que existem em todas as 
linguagens de programação, e que recebe o nome de \textbf{\textit{condicionais}}.\par

Elas permitem que a partir de determinada condição que deve ser analisada,
executar-
mos uma determinada ação, ou outra. Um exemplo de código javascript 
que possui essa determinada estrutura, é a seguinte:

\begin{verbatim}
  if(condição){
    // codigo executado se a condição for verdadeira
  }else{
    // código executado se a condição for falsa
  }
\end{verbatim}

A priori, temos que as palavras \texttt{if} e \texttt{else} são reservadas dentro do 
javascript para a construção de estruturas condicionais, portanto
temos que dentro dos parênteses, temos a condição que deve ser analisada, 
e dentro das chaves, os blocos de códigos a serem executados condicionalmente.

A posteriori, temos que as vezes as comparações condicionais de dois fatores
podem não ser suficientes para nós conseguirmos analisar o código e executá-lo 
da maneira desejada, portanto, temos a necessidade de comparar um determinado valor 
a várias possibilidades as quais ele pode assumir, e assim surge o \texttt{else if}:

\begin{verbatim}
  if(primeiraCondicao){
    // executar código do primeiro bloco
  }else if(segundaCondicao){
    // executar o segundo bloco de código
  }else{
    // executar bloco de código quando condições não forem atingidas
  }
\end{verbatim}

\subsection{Aninhando condicionais, e estabelecendo condições} % (fold)
\label{sub:Aninhando condicionais, e estabelecendo condições}
Temos que as condições presentes analisandas dentro das condicionais,
devem assumir valores booleanos, ou seja, de verdadeiro ou falso. Entretanto,
podemos criar análises numéricas, decimais, e de outros tipos primitivos, que 
possuem como resultado, um valor booleano, e inserir como a condição que controla
o fluxo da condicional, como no seguinte caso:

\begin{verbatim}
  let condicao = 15 < 20

  if(condicao){
    // executar código se a condicao for verdadeira
  }else{
    // executar código se a condicao for falsa

\end{verbatim}

Temos no exemplo anterior, que uma comparação lógica entre dois elementos numéricos
gerou um valor booleano, e isso pode ser feito de várias maneiras por meio dos operadores lógicos, ou por meio de operadores de comparação. Os operadores que possibilitam isso, são os de comparação matemática, e operadores booleanos, que são o \texttt{\&\&} e o \texttt{||}, que são de disjunção e conjunção respectivamente

Podemos também aninhar condicionais, ou seja, dentro de alguma determinada condi-
cional, fazer outra condicional que pode comparar o mesmo valor, ou um valor que não possua
conexão com o primeiro comparado.
% subsection Aninhando condicionais, e estabelecendo condições (end)

\subsection{Instrução switch} % (fold)
\label{sub:Instrução switch}
As isntruções condicionais são muito úteis, entretanto, quando temos uma variável
que pode assumir diversos valores, isso cria uma cadenia de condicionais que é dificil
de manter no código, e difícil de entender, portanto, é recomendando que quando 
comparamos condicçoes que podem assumir diversos valores, que usemos 
a estrutura \texttt{switch}.

\begin{verbatim}
  switch(expression){
    case choice1:
      // run code and break the comparation
      break;
    case choice2:
      // run code and break the comparation
      break;
    default:
      // only need to run the code actually
  } 
\end{verbatim}

Nesse caso, temos que dois casos existem, e que quebramos a comparação como última 
instrução do caso o qual o mesmo corresponde. Isso se deve ao fato de que o switch case 
pode desejar que uma expressão, encaixe em mais de um caso, sendo assim, o mesmo deve 
passar pelos dois casos, e executar os dois blocos de isntrução, e isso é feito não 
quebrando a comparação dos casos, ou seja, não adicionando a instrução \texttt{break}.
Caso nenhum dos casos seja verdadeiro, ou caso o que seja, não possua a instrução de 
quebrar a comparação, e seja o único verdadeiro, a expressão vai cair no caso padrão, 
e após executar as instruções, vai quebrar a comparação por ser o último caso 
propositadamente.
% subsection Instrução switch (end)

\subsection{Operador ternário} % (fold)
\label{sub:Operador ternário}
Temos que as comparações são elementos essenciais dentro da programação, entretanto 
elas ocupam muito espaço visual, e algumas comparações são tão pequenas que são quase
"banais", pra isso existe o operador ternário, que permite uma comparação de um 
elemento, que pode receber um de dois valores se sua condição for verdadeira, e essa 
instrução é feita toda em um única linha.

\begin{verbatim}
  (condição) ? se a condição for verdadeira : se a condição for falsa;
\end{verbatim}
% subsection Operador ternário (end)

\section{Loops} % (fold)
\label{sec:Loops}

% section Loops (end)

% section Cosntruindo elementos com o javascript (end)
\end{document}




