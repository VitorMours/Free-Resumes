\documentclass[12pt a4paper]{paper}
\usepackage[utf8]{inputenc}
\usepackage{multicol}
\title{Estudo sobre o PHP}
\author{1221194773 - João Vitor Rezende Moura}
\date{12/2023}

\begin{document}

\maketitle

\section{Introdução}
PHP(Preprocessador de Hypertexto) é uma linguagem de programação amplamente utilizada para desenvolvimento web, 
principalmente dentro do HTML, embutido, e é uma linguagem de servidor, que também pode ter outros usos por ser de 
propósito geral. Um exemplo de código PHP é o seguinte:

\begin{verbatim}
<!DOCTYPE html>
<html>
  <head>
    <title>Página estudando php</title>  
  </head>

  <body>
    <?php
      echo "Hi, im a script!";
    ?>
  </body>
</html>
\end{verbatim}

Em vez de um monte de comandos, podemos usar o PHP para usar código HTML, e dentro dele colocar o php embutido, para que 
assim ele faça o processamento das informações igualmente, mas de forma muito mais arrumada, e eficiente.
PHP é diferente de outras linguagens, e de tecnologias do client-side, como o JavaScript, porque ele é executado do 
lado do servidor. gerando o HTML completo que será enviado ao cliente. Podemos imclusive configurar o servidor web para 
processar todas as páginas html com php, e não será possível detectar sua presença pelo lado do usuário.


\subsection{Principais áreas de uso do php}

\subsection{Pré-requisitos}
Para podermos fazer uso do php, precisamos ter um servidor o qual suporte essa linguagem, para que o desenvolvimento de 
aplicações possa ser feito com ela de forma correta. Precisa-se instalar um web server com \textbf{Apache}, php e de 
banco de dados, costuma-se utilizar o \textbf{MySQL}. Essas ferramentas possibilitam rodar o php, e possibiltiar o 
armazenamento de dados que pode ser utilizado durante a construção da nossa aplicação, respectivamente.


\subsection{Primeira aplicação PHP}

\begin{verbatim}
<!DOCTYPE html>
<html>
    <head>
        <title>PHP Test</title>
    </head>
    <body>
        <?php echo '<p>Hello World</p>'; ?>
    </body>
</html>
\end{verbatim}

Esse código, permite que criemos um site HTML, e que dentro dele pré-processemos um código php que permite a criação de 
um parágrafo html, o arquivo deve possuir o nome de \texttt{hello.php}, e podemos com isso, acessar ele dentro do nosso 
navegador, que quando tiver o servidor rodando, pode ser acessado por meio de: \texttt{http://localhost/hello.php}



\section{Conceitos Iniciais}

\subsection{Tags PHP}
As tags do PHP, são a forma de indicarmos para o processador do arquivo o qual estamos escrevendo, onde está o fim e o 
início daquele determinado código php, pois elas rerpresentam a abertura e o fechamento, e são respectivamente
\texttt{<?php} e \texttt{?>}. O PHP também inclui uma tag encurtada, que pode ser utilizada quando o único comando o 
qual for utilizado dentro daquele determinado espaço é o echo, e essa tag se dá por \texttt{<?=}. Essas tags curtas estão habilitadas por padrão, mas podemos desabilitar elas por meio da diretira \textbf{short\_open\_tag}, presente dentro 
do arquivo de configuração \textbf{\textit{php.ini}} com a configuração \textbf{--disable-short-tags}

\subsection{Escapando HTML}
Tudo fora do par de tags do PHP é ignorado pelo interpretador do php, e é entendido como HTML, possiblitano assim a 
mistura dessas duas linguagens dentro do mesmo arquivo, oque pode ser muito útil na hora de criarmos templates. Isso fará
com que o interpretador comece a repassar o conteúdo que encontre posteriormente à tag de fechamento, como html, a menos 
que a instrução passada dentro do PHP seja uma condicional, como no exemplo a seguir:

\begin{verbatim}
<?php if ($expression == true): ?>
  Isso irá aparecer se a expressão for verdadeira.
<?php else: ?>
  Senão isso irá aparecer.
<?php endif; ?>
\end{verbatim}


\subsection{Separação de Instruções}
Como em outras linguagens da mesma época que o PHP foi criado, as instruções devem ser divididas entre si por meio de 
ponto e vírgula. A tag de fechamentod de um bloco de código PHP implica automaticamente em um ponto e vírgula.
Um tipo especial de instrução são os comentários, os quais permitem que nos escrevamos informações úteis dentro do código
, mas que não necessariamente vao ser processadas pelo interpretador e serem mostradas ao usuário, e esses comentários 
podem ser feitos de três formas:

\begin{itemize}
    \item // - Comentários de linha única
    \item /**/ - Comentários multilinha
    \item \# - Comentários de linha única
\end{itemize}

\subsection{Tipos primitivos dentro do PHP}
Os tipos primitivos são os tipos de dados que alguma variável ou informação dentro do código PHP pode assumir para si, 
sendo os tipos primitivos da linguagem os seguintes:


\begin{multicols}{3}
\begin{itemize}
    \item null
    \item bool
    \item int
    \item float
    \item string
    \item array
    \item object
    \item callable
    \item resource
\end{itemize}
\end{multicols}

O PHP é linguagem com tipagem dinâmica, ou seja, por padrão não existe a necessidade de especificar o tipo da variável,
 já que isso será determinado em tempo de execução. Entretanto, é possível de restringir estaticamente os tipos através 
 da declaração desses tipos primitivos. Essas tipagens restrigem determinadas operações de serem feitas dentro do nosso 
 código. Entretanto, uma expressão ou uma variável utilizada numa operação que o tipo não dê suporte, faz com que o 
 interpretador tente transformar o valor em algum que permite a solução da operação.

Para verificar o valor e o tipo de alguma determinada expressão, podemos usar a função \texttt{var\_dump()}, e para
extrair o tipo de um variável, podemos usar \texttt{get\_debug\_type()}.

\begin{verbatim}    
<?php
$a_bool = true;   // um valor boleano
$a_str  = "foo";  // um texto
$a_str2 = 'foo';  // um texto
$an_int = 12;     // um inteiro

echo get_debug_type($a_bool), "\n";
echo get_debug_type($a_str), "\n";

// Se essa variável conter um inteiro, aumento o número por quatro
if (is_int($an_int)) {
    $an_int += 4;
}
var_dump($an_int);

// Se $a_bool for um texto, imprima
if (is_string($a_bool)) {
    echo "String: $a_bool";
}
?>
\end{verbatim}

As variáveis podem ter comportamentos e atributos específicos conforme o tipo primitivo o qual elas pertencem, 
sendo assim, podemos entender de uma determinada maneira para cada tipo que elas podem possuir

\subsubsection{Características e atributos dos tipos primitivos}
Cada tipo primitivo possui sua peculiaridade específica, que pode ou não ser compartilhada com outros tipos, e temos 
que são:

\begin{itemize}
    \item \textbf{null:} É o tipo unitário do PHP, pode apenas assumir o valor de nulo, e ele é designado para variáveis indefinidas e variáveis que passam pelo método \texttt{unset()}. 
    \item \textbf{booleanos:} São variáveis que representam valores lógicas, ou seja, de verdadeiro ou falso.
    \item \textbf{inteiros:} São variáveis as quais permite guardamos valores numéricos que não possuem casas decimais
    \item \textbf{float:} São variáveis que assim como os inteiros, guardam valores numéricos, mas nesse tipo específico, elas podem asumir uma parte que é decimal dentro da sua informação.
    \item \textbf{string:} É uma série de caracteres o qual um caractere é representado por um byte. Atualmente dentro do php, podem ser utilizados os dois tipos de aspas para definir uma string, e além disso, temos a sintaxe heredoc e temos a sintaxe nowdoc.
    \item \textbf{strings numéricas:} Strings numéricas são strgins que podem ter seu conteúdo   
    \item \textbf{array:} Um arrai pode ser entendido como um mapa ordenado, que define relação entre valores e suas respectivas chaves
    \item \textbf{object:} São objetos, que são elementos presentes do paradigma orientado à objetos 
    \item \textbf{callable:} 
    \item \textbf{resource:} É uma variável especial que mantém uma referência a um recurso externo 
\end{itemize}

\subsubsection{Valores lógicos e numéricos} % (fold)
\label{sec:Valores lógicos e numéricos}

\paragraph{Valores Booleanos: } % (fold)
\label{par:Valores Booleanos: }
Dentro do PHP, podemos ter valores lógicos, ou seja, determinar uma variável com valor 
de verdadeiro ou falso, comparar alguma variável com esse valor, ou modificar 
expressões lógicas por meio desses valores, uma forma de ver isso dentro do PHP, é da 
seguinte forma:

\begin{verbatim}
<?php
  if($action == "mostrar_versao"){
    echo "Versão 1.23";
  }
  if($exibir_comentários == TRUE){
    echo "<hr>\n";
  }
?>  
\end{verbatim}

Nesse código podemos ver, que uma determinada parte dele compara o valor de uma 
vaiável com um valor lógico booleano, vendo se é verdadeiro ou falsa necessidade de 
exibir os comentários ad página. Entretanto, vemos que a estrutura condicional verifica 
se uma comparação é verdadeira, com isso, não precisamos necessariamente fazer essa 
comparação, podemos apenas colocar dentro da condicional o valor da variável em si. 

\paragraph{Valores Inteiros:} % (fold)
\label{par:Valores Inteiros:}
Os valores inteiros são valores numéricos normais, os quais podem ser usados para 
representar grandezas positivas ou negativas, ele pode especificar um valor de notação 
decimal, hexadecimal e octal, além da base binária. Para poder usar essas notações 
especiais, elas precisam ser especificadas durante a delacração das variáveis, da 
seguinte forma:

\begin{verbatim}
<?php
  $a = 123;
  $b = 0o123;
  $c = 0x1A;
  $d = 0b11111111;
  $e = 1_234_567;   
?>  
\end{verbatim}

Em caso de números inteiros muito grandes, pode acontecer de surgi um overflow de
números inteiros, oque transforma o valor inteiro em um valor flutuante automaticamente.

\paragraph{Conversão para valores inteiros} % (fold)
\label{par:Conversão para valores inteiros}
Para converter os valores para números inteiros, temos que usar o comando \texttt{(int)}
, também pode ser usada a função \texttt{intval()} para transformar o valor em inteiro. 

\begin{verbatim}  
  $a = ((int) False); // Convertendo o valor de booleano para decimal, nesse caso 0
  $b = ((int) 8.1);
  $c = ((int) 8.0);
\end{verbatim}
% paragraph Conversão para valores inteiros (end)


\paragraph{Números de Ponto Flutuante} % (fold)
\label{par:Números de Ponto Flutuante}
Números de ponto flutuante podem ser entendidos como números decimais dentro do mundo 
matemático. Eles possuem uma precisão limitada pelo compilador da linguagem quando usad 
para valores muito pequenos.
% paragraph {Números de Ponto Flutuante} (end)
% subsubsection Valores lógicos e numéricos (end)

\subsubsection{Strings} % (fold)
\label{sec:Strings}
Uma string pode ser considerada um conjunto de carecteres de que são armazenados de 
forma sequêncial, sendo cada um deles representado por um byte específico. Temos a 
construção das strings feita por meio de quatro meios, sendo eles:

\begin{itemize}
  \item aspas simples
  \item aspas dulpas
  \item sintexe heredoc 
  \item sintexe nowdoc
\end{itemize}

As aspas simples, diferente das outras formas não possui escape de caracteres 
especiais quando ocorrem dentro de sua delimitação.

\paragraph{Heredoc} % (fold)
\label{par:Heredoc}
O heredoc é um operador, que funciona como um identificado que é seguido por uma nova 
linha. A própria string é colocada em seguida e a seguir o mesmo identificador é usado 
para finalizar. 

\begin{verbatim}
echo <<<END 
    a
  b
c
END;
\end{verbatim}

% paragraph Heredoc (end)


\paragraph{Nowdoc} % (fold)
\label{par:Nowdoc}

% paragraph Nowdoc (end)











% subsubsection Strings (end)

\end{document}

